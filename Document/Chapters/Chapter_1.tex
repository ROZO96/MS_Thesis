Throughout history, the field of astronomy and astrophysics, among other scientific fields, has relied on two main methods to gain knowledge: empirical observation and experimentation, and logical reasoning and theorizing. Astronomers observe the universe and meticulously record their findings of the night sky, while astrophysicists use their knowledge to create hypotheses and make predictions about what they expect to observe. However, due to the vast amount of complex physical processes that occur in astrophysical scenarios, and the difficulty of creating fully controlled experiments to test theoretical predictions, a new approach was needed to generate knowledge within these fields of science.\\
Hence, Computer simulations have appeared as a modern way to create knowledge in astrophysics (as well as in other branches of science and engineering). These numerical experiments enable physicists to set up different physical scenarios and study how they evolve over time and space under a certain set of equations/models. In that sense, it is expected that if the relevant physical laws were taken into account the results obtained from the simulation should closely resemble empirical observations.\\



adiooosss	
