\documentclass[12pt,a4paper]{report}

\usepackage{graphics}
\usepackage{fullpage,epsf,graphicx}%, amstext,url} 
\usepackage{physics,amsmath}

\def\BibTeX{{\rm B\kern-.05em{\sc i\kern-.025em b}\kern-.08em
    T\kern-.1667em\lower.7ex\hbox{E}\kern-.125emX}}
\bibliographystyle{unsrt}  
\begin{document}

\thispagestyle{empty}

%
%	This is a basic LaTeX Template for the TP/MP MSc Dissertation report

\parindent=0pt          %  Switch off indent of paragraphs 
\parskip=5pt            %  Put 5pt between each paragraph  

%	This section generates a title page
%       Edit only the sections indicated to put in the project title, and submission date

\vspace*{0.1\textheight}

\begin{center}
        \huge{\bfseries Title of my Dissertation}\\ % Replace with the title of your dissertation!
\end{center}

\medskip

\begin{center}
        \Large{Daniel Rozo}\\  % Author of dissertation - replace with your name!
        \medskip
        \large{August 18, 2023}  % Submission date
\end{center}

%%% If necessary, reduce the number 0.4 below so the University Crest
%%% and the words below it fit on the page.
%%% Don't let the crest, or the wording below it, flow onto the next page!

\vspace*{0.4\textheight}

\begin{center}
        \includegraphics[width=35mm]{Figures/crest.pdf}
\end{center}

\medskip

\begin{center}

%%%
%%% Change Theoretical to Mathematical if appropriate
%%%
\large{
  MSc in Theoretical Physics\\[0.8ex]
  The University of Edinburgh\\[0.8ex]
  2023
}

\end{center}

\newpage


\pagenumbering{roman}

\begin{abstract}
This is where you summarise the contents of your dissertation. It should be
at least 100 words, but not more than 250 words.
\end{abstract}

\pagenumbering{roman}

\begin{center}
\textbf{Declaration}
\end{center}

I declare that this dissertation was composed entirely by myself.

Chapters 2 and 3 provide an introduction to the subject area and a
description of previous work on this topic. They do not contain
original research.

Chapter~4 describes work that was done entirely by me. The results of
this chapter have been obtained previously by Anne T Matta of the
University of Kinlochteuchter, but the methods used here are different
in some important (or minor) ways.

Chapters 4 through 6 contain my original work. The work described in
Chapter~4 was done in collaboration with Professor Carole Ann O'Malley
and her PhD student Jake O'Bean. Chapter~5 presents original work done
entirely by me.

\bigskip

State whether calculations were done using Mathematica, Maple, MATLAB,
SymPy, etc, with (or without) gamma-matrix manipulation code, master
integrals, the Super-Duper software package, etc. In other words, you
should refer to any software that you used during your project. For
example, Monte Carlo simulation packages, hydrodynamics packages,
measurement code, fitting code, tensor algebra and/or calculus packages,
Feynman diagram evaluation packages, etc.

State whether any software you used was written by you from scratch,
by your supervisor (or by whoever), or if it's a standard package.

\newpage

\begin{center}
\textbf{Personal Statement}
\end{center}

\emph{You \textbf{\emph{must}} include a Personal Statement in your
  dissertation. This should describe what you did during the project,
  and when you did it. Give an account of problems you faced and how
  you attempted to overcome them. The examples below are based on
  personal statements from MSc and MPhys projects in previous years,
  with (mostly-obvious) changes to make them anonymous. }

\subsubsection*{Example~1: an analytical project}

The project began with an introduction to the spinor-helicity
formalism in four dimensions, with my main source material being
H. Elvang's “Scattering Amplitudes in Gauge Theory and Gravity” [1]. I
read the first chapter, and acquainted myself with the formalism,
and how it worked in a practical sense.

Once I felt more comfortable with it, we moved onto the
six-dimensional spinor-helicity formalism paper, where I spent some
time gaining as strong an understanding of how the formalism worked,
and proving identities.

The next stage was to learn about the generalised unitarity procedure,
with the end goal being to use it to calculate coefficients for some
one loop integral, likely involving massive particles. Learning how
this worked took some time, and proved to be some of the most
difficult material for me to understand.

It wasn't until later that we began to consider applying what I had
learned to a Kaluza-Klein reduction, which ended up being the main
focus of the project. It mixed well with the general theme of
“extra-dimensional theory” the project began with, and allowed me to
apply all that I'd learned and prepared for so far.  The vast majority
of my remaining time was spent calculating coefficients for the scalar
box contribution to the gluon-gluon to two-Kaluza-Klein-particle
amplitude, overcoming a number of problems and errors, to finally have
human-readable, and presentable results.

During the course of the project, I met with my supervisor every week,
in order to discuss my progress and the direction I would head
next. Toward the end, the frequency of our meetings increased
somewhat, as I began to finish my calculations.

I started writing this dissertation in mid-July, and I spent the first
three weeks of August working on it full-time.

Overall, I feel that the project was a success, and I found it to be
extremely enjoyable throughout.


\subsubsection*{Example~2: a computational project}

I spent the first 2 weeks of the project reading the material
surrounding my project - mainly [1] and [2]. I also began to plan out
how I would implement the algorithms in C++, in doing this I gained an
understanding of what the main goals of the first half of my project
would be and how they could be achieved. I identified which Monte
Carlo observables would be useful to measure in these simulations.

For the next 3 weeks I implemented the standard Atlantic City
algorithm and debugged my code whilst developing analysis tools in
python. I compared the results from my simulations to the results from
[3] (for the Random Osculator) and [4] for the EvenMoreRandom
Osculator. Having obtained positive results for the Random Osculator I
started reading up on Heaviside Articulation. I examined how to
integrate a Heaviside Articulator into the simulation in order to
produce the most efficient simulation - the solution I decided on was
to use a package called HeaviArt[5].

Following this I began to integrate the Heaviside Articulator into my
code and test it against the regular algorithm. In addition to this I
ran longer simulations to verify my findings without Articulation.

In mid July I finished implementing Heaviside Articulation into my
code and began looking into how to quantify any improvement in speed
gained by this algorithm. As July progressed I started looking into
how to integrate the EvenMoreRandom Osculator into my code - this was
the most complicated part of the project, as discussed in the body of
this report. Despite much effort on my part, I couldn't get the
results produced by the new algorithm to agree with the old
ones. Following further study of the literature, and long discussions
with Jack O'Bean, it turned out that the original form of Heaviside
Articulation didn't applied to the EvenMoreRandom Osculator. With the
help of Jack and my supervisor, I then developed the new version
described in this report. I also did analytical calculations of the
four-point blue function to two orders higher than had
been published previously in the literature.

For the final parts of the summer I worked mainly on perfecting the
algorithm for the Random Osculator and implementing the EvenMoreRandom
Osculators algorithm with the improved Heaviside Articulation. The
final results were encouraging, but more work is clearly needed. To
this end, I have been awarded a studentship by the British University
of Lifelong Learning to extend this work during my PhD Studies at
the non-existent Scottish Highlands Institute of Technology in
Inveroxter.

I started writing this dissertation in mid-July, and I spent the first
three weeks of August working on it full-time.


\subsubsection{Example~3: a more mathematical project}

My first two weeks of work on the project consisted of building up a
working knowledge of the algebraic structures and techniques which
would be used in the main calculations which were to be carried out,
in particular carefully reading up on the 11-dimensional case [1, 2],
the general structure of the Spencer complex of the Poincar\'e
superalgebra, the Clifford and exterior algebras over a Lorentzian
vector space in general finite dimension and the relationship between
them, the spin group, the Lorentz algebra $so(V$) and their vector and
spinor representations. As a toy calculation, I computed the first two
Spencer cohomology groups of the Poincar\'e algebra and its relevant
subalgebras in order to find its filtered (sub)deformations, finding
that these are given by a space of algebraic curvature operators. This
calculation is vastly simplified compared to the superalgebra
calculation owing to the lack of spinor structure.

In the following two weeks, I turned to the particular case of 5
dimensions, initially study- ing the quaternionic structure of the
relevant Clifford algebra, its spinor representation and symplectic
Majorana spinors. I read up on complex and quaternionic structures and
representations, proved various identities for products of higher-rank
gamma matrices found in [5] and derived the Fierz identity and various
other useful identities involving quantities defined in terms of the
Majorana spinors. I made an initial attempt to solve the relevant
cocycle conditions, initially finding that the space of solutions was
given by 3-forms (or Hodge-dually 2-forms), but from the known spinor
connection in D = 5 supergravity we knew that there was a problem with
the solution. Working with my supervisor to fix the errors in this
calculation, finding the most efficient way of setting it out and
writing up the work done so far and some of the background material
took up the following two to three weeks.  Next, I moved on to
figuring out the fundamentals of the geometric part of the project,
understanding spin structures on manifolds, the spin lift of the
Levi-Civita connection, the spinorial Lie derivative and the
definition of the superconnection. I derived the flatness conditions
for the curvature of the superconnection and showed that these
conditions are sufficient to cause the Killing superalgebra to close
and form a Lie superalgebra. This work, along with further writeup,
took another two weeks.

The final weeks of the project were spent learning about Lorentzian
and Riemannian symmetric spaces, finding all of the the maximally
supersymmetric backgrounds, on the way learning a little about the
supersymmetric solutions of 5-dimensional supergravity, and finally
finishing the dissertation.




\newpage

\begin{center}
%\vspace*{2in}
% an acknowledgements section is completely optional but if you decide
% not to include it you should still include an empty {titlepage}
% environment as this initialises things like section and page numbering.
\textbf{Acknowledgements}
\end{center}

\emph{Put your acknowledgements here. Thanking your supervisor for
his/her help is standard practice, but it's not compulsory \ldots}

I'd like to thank my supervisor Professor Carole Ann O'Malley for
making this project possible, and her PhD student Jack O'Bean for his
patience and his detailed functional explanations of how classical
symmetries can be broken by quantum effects. Thanks also to Wally Bee
and Ken Garoo of the University of Woolloomooloo for sending me
their hopping-parameter expansions.

Finally, none of this would have been possible without financial
support from Paterson's Lane Education Committee.

\bigskip

This document has its origins in the dissertation template for the MSc
in High Performance Computing, which is apparently descended from a
template developed by Professor Charles Duncan for MSc students in
Meteorology. His acknowledgement follows:

\emph{This template has been produced with help from many former
  students who have shown different ways of doing things. Please make
  suggestions for further improvements.}

Some parts of this template were lifted unashamedly from the Edinburgh
MPhys project report guide, with little or no modification. I have no
idea who wrote the first version of that\ldots

You don't have to use \LaTeX\ for your dissertation. You can use
Microsoft Word, Apple Pages, LibreOffice (or similar) if you prefer,
but it's \emph{much} easier to typeset equations in \LaTeX, and
references look after themselves. Whatever you use, your dissertation
should have the general structure of this template, and it should look
similar -- especially the front page.

\tableofcontents
\listoftables
\listoffigures

\pagenumbering{arabic}

\chapter{Introduction}
Throughout history, the field of astronomy and astrophysics, among other scientific fields, has relied on two main methods to gain knowledge: empirical observation and experimentation, and logical reasoning and theorizing. Astronomers observe the universe and meticulously record their findings of the night sky, while astrophysicists use their knowledge to create hypotheses and make predictions about what they expect to observe. However, due to the vast amount of complex physical processes that occur in astrophysical scenarios, and the difficulty of creating fully controlled experiments to test theoretical predictions, a new approach was needed to generate knowledge within these fields of science \cite{Sundberg2010CulturesAstrophysics}.\\
Hence, Computer simulations have appeared as a modern way to create knowledge in astrophysics (as well as in other branches of science and engineering). These numerical experiments enable physicists to set up different physical scenarios and study how they evolve over time and space under a certain set of equations and models \cite{Sundberg2012CreatingAstrophysics}. In that sense, it is expected that if the relevant physical laws were taken into account the results obtained from the simulation should closely resemble observations \cite{Heng2014TheSimulations}. Comparing these models' predictions with empirical data enable us to verify, improve and modify it to increase the agreement with observation and hence gain a deeper understanding of the studied astrophysical phenomena \cite{Muller2005SimulatingAchievements}.

Over recent years, advancements in computer and processor technology have made it possible to study increasingly complex astrophysical scenarios through computational simulations. These scenarios include the cosmological evolution of the universe, the formation of stars and galaxies \cite{Vogelsberger2020CosmologicalFormation, Garrison-Kimmel2017NotGalaxies} as well as the study of black holes and supernovae \cite{Brown2008TurduckeningStudy, Donmez2006NumericalCode, Muller2004TowardModels}. The simulations use models that can describe the cooling of gas, radiation fields, magnetic fields, and relativistic particles among other physical phenomena.

In many of the situations previously described, baryonic (ordinary) matter plays a key role in the dynamics as also being part of the visible component of most of them, and it is mostly present in the gaseous phase as hydrogen and helium.  Gases in astrophysical simulations are usually modeled through Euler equations which consist of a set of Partial Differential Equations (PDEs) representing mass, momentum, and  energy conservation on an inviscid fluid continuum (See equation \ref{eq:Euler}). 

\begin{subequations}\label{eq:Euler}
\begin{gather}
    \pdv{\rho}{t} + \div{\rho \vb{u}}=0\\
    \pdv{\rho\vb{u}}{t} + \div(\rho \vb{uu} +p\vb{I})=0\\
    \pdv{e}{t} + \div{\rho h\vb{u}}=0
\end{gather}
\end{subequations}

Due to the non-linear nature of the systems of PDEs, it is only possible to find analytical solutions to them in very simple specific cases but to be able to solve Euler's equations in more general situations a computational/numerical approach is necessary. The numerical methods used to discretize hydrodynamics equations can be 
classified into three main categories: Eulerian, Lagrangian, and Arbitrary Eulerian-Lagrangian (ALE). 



\chapter{Background theory and/or theory}
\input{Chapters/Chapter_2}
\chapter{Design and/or development (of my project)}
This section should be written in standard scientific
language. Standard techniques in your research field should not be
written out in detail. In computational projects this section should
be used to explain the algorithms used and the layout of the
computational code. A copy of the actual code may be given in the
appendices if appropriate.

This section should emphasise the philosophy of the approach used and
detail novel techniques. However please note: this section should not
be a blow-by-blow account of what you did throughout the project. It
should not contain large detailed sections about things you tried and
found to be completely wrong! However, if you find that a technique
that was expected to work failed, that is a valid result and should be
included.


Here logical structure is particularly important, and you may find
that to maintain good structure you may have to present the
explorations/calculations/computations/whatever in a different order
from the one in which you carried them out.

\chapter{Results and Analysis}
\chapter{Conclusions}
\bibliography{References/references.bib} 
\appendix
\chapter{Stuff that's too detailed}
\include{Chapters/Appendix_A}





\end{document}

